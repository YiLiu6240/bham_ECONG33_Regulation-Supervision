\documentclass[12pt]{article}
\usepackage[utf8]{inputenc}
\usepackage[]{geometry}
\usepackage{setspace}
\usepackage{pdflscape}
\usepackage{amsmath}
\usepackage{amsfonts}
\usepackage{amssymb}
\usepackage{sectsty}
\usepackage[titles,subfigure]{tocloft} % Alter the style of the Table of Contents
\usepackage[titletoc,toc,title]{appendix}
\usepackage{tabulary}
\usepackage{booktabs} % for much better looking tables
\usepackage[]{threeparttable}
\usepackage{array} % for better arrays (eg matrices) in maths
\usepackage{verbatim} % adds environment for commenting out blocks of text & for better verbatim
\usepackage{subfig} % make it possible to include more than one captioned figure/table in a single float
\usepackage{longtable}
\usepackage{multirow}
\usepackage{graphicx} % support the \includegraphics command and options
\usepackage{placeins} % clear floats
\usepackage[round]{natbib}
\usepackage{hyperref}

\hyphenpenalty=5000 % reduce hyphenation
\tolerance=1000
\doublespacing
\renewcommand\bibname{References}


\author{Yi Liu (yxl105[at]bham.ac.uk)\\\scriptsize{\url{https://github.com/YiLiu6240/bham_ECONG33_Regulation-Supervision}}}
\date{\today}
\title{G33D - Class Note 1\\Financial Statements}

\begin{document}
\maketitle

\section*{Questions One - Six}

Please refer to \citet[Chap.2]{hull2012risk} for the discussions of the activities of banks, the financial statements of banks, and different risks.
You need to know how the various operations of a bank are reflected in the different items of the financial statements, and how gains and losses are reflected as well.

\section*{Question Seven}

In order to find the ``additional'' capital to hold, we need to find a value \(V\) in the distribution of income so that:

``We will be 99.9\% sure that we will not get an income less than V%
\footnote{See \citet[Chap.9, pp.183-185]{hull2012risk} for explanations.}.''

Or equivalently,

\begin{align*}
  Pr(X \leq V) & = 100\% - 99.9\% \\
               & = 0.1\%
\end{align*}

where \(X\) denotes a random outcome in the distribution of
\(X \sim N(\mu, \sigma^2)\), \(\mu = 0.6, \sigma = 2\)
with a significance level \(\alpha = 0.001\).

So \(V\) will be the \(0.001^{th}\) quantile (\(0.1^{th}\) percentile) value in the specified normal distribution:%
\footnote{
  \(Z_{\alpha} = -3.090\)  when  \(\alpha = 0.001\).
  You need to refer to a statistics book about the transformation between a standardised normal critical value and a normal critical value.}%
:

\begin{align*}
  V & = \mu + Z_{\alpha} \times \sigma \\
    & = 0.6 + (- 3.09)*2 \\
    & = -5.58
\end{align*}

Let \(C\) be the additional capital level required, and we have:

\begin{align*}
  5 + C + V \geq 0, \Rightarrow C \geq 0.58
\end{align*}

Therefore regulators will require the bank to hold additional \$0.58 million capital to guarantee that its capital will not be wiped out for 99.9\% of the situation.

\bibliographystyle{jmr}
\bibliography{G33D_bib}

\end{document}
