\documentclass[12pt]{article}
\usepackage[utf8]{inputenc}
\usepackage[]{geometry}
\usepackage{setspace}
\usepackage{pdflscape}
\usepackage{amsmath}
\usepackage{amsfonts}
\usepackage{amssymb}
\usepackage{sectsty}
\usepackage[titles,subfigure]{tocloft} % Alter the style of the Table of Contents
\usepackage[titletoc,toc,title]{appendix}
\usepackage{tabulary}
\usepackage{booktabs} % for much better looking tables
\usepackage[]{threeparttable}
\usepackage{array} % for better arrays (eg matrices) in maths
\usepackage{verbatim} % adds environment for commenting out blocks of text & for better verbatim
\usepackage{subfig} % make it possible to include more than one captioned figure/table in a single float
\usepackage{longtable}
\usepackage{multirow}
\usepackage{graphicx} % support the \includegraphics command and options
\usepackage{placeins} % clear floats
\usepackage[round]{natbib}
\usepackage{hyperref}

\hyphenpenalty=5000 % reduce hyphenation
\tolerance=1000
\doublespacing
\renewcommand\bibname{References}


\author{Yi Liu (yxl105[at]bham.ac.uk)\\\scriptsize{\url{https://github.com/YiLiu6240/bham_ECONG33_Regulation-Supervision}}}
\date{\today}
\title{G33D - Class Note 2\\Value-at-Risk}

\begin{document}
\maketitle

\section*{Value at Risk}

(For detailed discussion of Value at Risk please refer to
\citet[Chap.9]{hull2012risk})

In this class set we have come across the calculations of $ VaR $s of different situations:
outcomes (gains/losses) being continuously distributed, outcomes being discretely distributed, the potential non-uniqueness of $ VaR $s, portfolios and different times horizons.

Since the definition of $ VaR $ shares some common feature with the definition of the cumulative distribution function of a random variable (be careful with the mathematical notation when you define the distribution as gains or losses!), so we can find the $ VaR $ from the CDF graph when constructed correctly.
In the case of a continuous random variable, a probability density function might give you a more intuitive idea about $ VaR $ as a certain critical value in the distribution graph.
You need to know what the CDF and PDF of a random variable will be in different cases (especially the CDF of a discrete variable and how to interpret it!).

In terms of portfolios and time-horizons, you need to be careful about how outcomes of each single item are correlated.
If we have independent projects in a portfolio, we can work out the outcomes of the portfolio from the outcomes of single projects and their corresponding probabilities, and then find the $ VaR $ from the CDF graph of the portfolio. If the outcomes are identically independently distributed, we
can calculate the $ VaR $s for different time horizons from the formula we discussed in class%
\footnote{
  If you are curious about how the formula comes from, there is no straight
  derivation in \textit{Hull}, however we can get the formula from the general formula of aggregating $ VaR $s \citet[pp. 187]{hull2012risk} when outcomes of each single items in a portfolio are correlated in a certain way.
  If we think of the $ VaR $ of a certain time horizon as the $ VaR $ of a portfolio, which we know the $ VaR $ of a single period, we can derive the formula when the distribution is $ IID $.
}.
For a general discussion of $ VaR $s when outcomes of each single item are correlated, please read
\citet[Chap.9, pp.187]{hull2012risk}
or
\citet[pp.200]{MatthewsThompson2008Banking}%
.

\bibliographystyle{jmr}
\bibliography{G33D_bib}

\end{document}
