\documentclass[12pt]{article}
\usepackage[utf8]{inputenc}
\usepackage[]{geometry}
\usepackage{setspace}
\usepackage{pdflscape}
\usepackage{amsmath}
\usepackage{amsfonts}
\usepackage{amssymb}
\usepackage{sectsty}
\usepackage[titles,subfigure]{tocloft} % Alter the style of the Table of Contents
\usepackage[titletoc,toc,title]{appendix}
\usepackage{tabulary}
\usepackage{booktabs} % for much better looking tables
\usepackage[]{threeparttable}
\usepackage{array} % for better arrays (eg matrices) in maths
\usepackage{verbatim} % adds environment for commenting out blocks of text & for better verbatim
\usepackage{subfig} % make it possible to include more than one captioned figure/table in a single float
\usepackage{longtable}
\usepackage{multirow}
\usepackage{graphicx} % support the \includegraphics command and options
\usepackage{placeins} % clear floats
\usepackage[round]{natbib}
\usepackage{hyperref}

\hyphenpenalty=5000 % reduce hyphenation
\tolerance=1000
\doublespacing
\renewcommand\bibname{References}


\author{Yi Liu (yxl105[at]bham.ac.uk)\\\scriptsize{\url{https://github.com/YiLiu6240/bham_ECONG33_Regulation-Supervision}}}
\date{\today}
\title{G33D - Class Note 2\\Value-at-Risk}

\begin{document}
\maketitle

\section*{Concept}

(For detailed discussion of Value-at-Risk please refer to \citet[Chap.9]{hull2012risk}).

What we try to get from Value-at-Risk is to measure the maximum level of losses given a confidence level, or ``how bad can the losses be for p\% of all the possible scenarios''.

Using the scenario we discussed in Class set 1, where the bank's profit follows a normal distribution of \(X \sim N(\mu = 0.6, \sigma^2 = 2^2)\). We try to pinpoint a \(V\) which divides the area of probabilities into two portions \(0.1\%\) and \(99.9\%\).

\begin{figure}[h]
  \centering
  \caption{\(VaR\) in a Normal Distribution}
  \begin{tabular}{cc}
  \includegraphics[width=0.4\textwidth]{output/note2_normal_pdf.pdf} &
  \includegraphics[width=0.4\textwidth]{output/note2_normal_cdf.pdf} \\
  \(pdf\) & \(cdf\)
  \end{tabular}
  \label{fig:note2_normal_pdf}
\end{figure}

\begin{figure}[h]
  \centering
  \caption{\(VaR\) measured as losses}
  \begin{tabular}{cc}
  \includegraphics[width=0.4\textwidth]{output/note2_normal_pdf_losses.pdf} &
  \includegraphics[width=0.4\textwidth]{output/note2_normal_cdf_losses.pdf} \\
  \(pdf\) & \(cdf\)
  \end{tabular}
  \label{fig:note2_normal_pdf}
\end{figure}

% In other words, we try to pinpoint an \(x\) (or a range of \(x\)) that is associated with the cumulative probability on the \(cdf\).

% The location of \(VaR\) is also dependent on how we denote \(x\). In the above example, we denote \(x\) as ``gains'', and when we denote \(x\) as ``losses'', we get:

% As we have discussed in the class tutorial, \(VaR\) is also applicable when dealing with discrete outcomes.
% Refer to XXX for more details.

\section*{Portfolio \(VaR\)}

\bibliographystyle{jmr}
\bibliography{G33D_bib}

\end{document}
