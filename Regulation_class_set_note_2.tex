\documentclass[12pt]{article}
\usepackage[utf8]{inputenc}
\usepackage[]{geometry}
\usepackage{setspace}
\usepackage{pdflscape}
\usepackage{amsmath}
\usepackage{amsfonts}
\usepackage{amssymb}
\usepackage{sectsty}
\usepackage[titles,subfigure]{tocloft} % Alter the style of the Table of Contents
\usepackage[titletoc,toc,title]{appendix}
\usepackage{tabulary}
\usepackage{booktabs} % for much better looking tables
\usepackage[]{threeparttable}
\usepackage{array} % for better arrays (eg matrices) in maths
\usepackage{verbatim} % adds environment for commenting out blocks of text & for better verbatim
\usepackage{subfig} % make it possible to include more than one captioned figure/table in a single float
\usepackage{longtable}
\usepackage{multirow}
\usepackage{graphicx} % support the \includegraphics command and options
\usepackage{placeins} % clear floats
\usepackage[round]{natbib}
\usepackage{hyperref}

\hyphenpenalty=5000 % reduce hyphenation
\tolerance=1000
\doublespacing
\renewcommand\bibname{References}


\author{Yi Liu (yxl105[at]bham.ac.uk)\\\scriptsize{\url{https://github.com/YiLiu6240/bham_ECONG33_Regulation-Supervision}}}
\date{\today}
\title{G33D - Class Note 2\\Value-at-Risk}

\begin{document}
\maketitle

\section*{Concept}

(For detailed discussion of Value-at-Risk please refer to \citet[Chap.9]{hull2012risk}).

What we try to get from Value-at-Risk is to measure the maximum level of losses given a confidence level, or ``how bad can the losses be for p\% of all the possible scenarios''.

Using the scenario we discussed in Class set 1, where the bank's profit follows a normal distribution of \(X \sim N(\mu = 0.6, \sigma^2 = 2^2)\), we try to pinpoint a \(V\) which divides the area of probabilities (under the probability density function \(pdf\) curve) into two portions \(0.1\%\) and \(99.9\%\).
It's even easier for us to spot \(V\) that is associated with a cumulative probability of \(0.001\) on the cumulative distribution function \(cdf\) plot. The \(V\) is the \(VaR\) we are looking for (Figure~\ref{fig:note2_VaR_gaussian}).

Under such a normal distribution setting, we calculate \(VaR\) from the ``transformation'' formula\footnote{
  which is derived from the normal distribution quantile function (inverse function of \(cdf\)):

  \begin{equation*}
    VaR_{1-\alpha} = F^{-1}_{X} (x) = \mu + \sigma \sqrt{2} {erf}^{-1} (2F - 1)
  \end{equation*}
}:
\begin{equation*}
  VaR_{1-\alpha} = \mu + Z_{\alpha} \sigma
\end{equation*}

\begin{figure}[h]
  \centering
  \caption{\(VaR\) in a Normal Distribution}
  \label{fig:note2_VaR_gaussian}
  \begin{tabular}{cc}
  \includegraphics[width=0.4\textwidth]{output/note2_normal_pdf.pdf} &
  \includegraphics[width=0.4\textwidth]{output/note2_normal_cdf.pdf} \\
  \(pdf\) & \(cdf\)
  \end{tabular}
\end{figure}

In addition, note that the \(VaR\) is represented differently if we denote \(x\) as losses (Figure~\ref{fig:note2_VaR_losses}).

\begin{figure}[h]
  \centering
  \caption{\(VaR\) measured as losses}
  \label{fig:note2_VaR_losses}
  \begin{tabular}{cc}
  \includegraphics[width=0.4\textwidth]{output/note2_normal_pdf_losses.pdf} &
  \includegraphics[width=0.4\textwidth]{output/note2_normal_cdf_losses.pdf} \\
  \(pdf\) & \(cdf\)
  \end{tabular}
\end{figure}

We discussed in the class tutorial how \(VaR\) can be obtained for other kinds of continuous distributions.
We also discussed how \(VaR\) can be applied when dealing with discrete outcomes.
Refer to \citet[p.186]{hull2012risk} for details.

\section*{Portfolio \(VaR\) \citet[p.192-194, p.195-197]{hull2012risk}}

From portfolio theory we derive the returns and volatility of an investment portfolio from the elements (weight \(w_i\), return \(R_i\), volatility \(\sigma_i\), correlation \(\rho_{i, j}\)) of individual investment:
\begin{align*}
  \mu_p & = E(R_p) \\
  R_p & = \sum_{i=1}^{n} w_i R_i \\
  \sigma_p^2 & = \sum_{i=1}^{n} w_i^2 \sigma_i^2
  + 2 \sum_{1 \leq i < j \leq n} \rho_{i, } w_i w_j \sigma_i \sigma_j
\end{align*}

Suppose we still retain the normal distribution assumption for the returns of individual investment in a portfolio, then we can derive \(VaR_{1-\alpha, p} = \mu_p + Z_{\alpha} \sigma\) by sustituting the individual elements into the equation.

The multi-period \(VaR\) can essentially be represented as a portfolio \(VaR\) where we should take autocorrelation into account.

If we further impose the condition that the individual returns are \(i.i.d\) then we can derive simpler expression for the portfolio/multi-period \(VaR\). However this strong condition is generally not held.

\bibliographystyle{jmr}
\bibliography{G33D_bib}

\end{document}
