\documentclass[12pt]{article} \usepackage[utf8]{inputenc}
\usepackage[]{geometry}
\usepackage{setspace}
\usepackage{pdflscape}
\usepackage{amsmath}
\usepackage{amsfonts}
\usepackage{amssymb}
\usepackage{sectsty}
\usepackage[titles,subfigure]{tocloft} % Alter the style of the Table of Contents
\usepackage[titletoc,toc,title]{appendix}
\usepackage{tabulary}
\usepackage{booktabs} % for much better looking tables
\usepackage[]{threeparttable}
\usepackage{array} % for better arrays (eg matrices) in maths
\usepackage{verbatim} % adds environment for commenting out blocks of text & for better verbatim
\usepackage{subfig} % make it possible to include more than one captioned figure/table in a single float
\usepackage{longtable}
\usepackage{multirow}
\usepackage{graphicx} % support the \includegraphics command and options
\usepackage{placeins} % clear floats
\usepackage[round]{natbib}
\usepackage{hyperref}

\hyphenpenalty=5000 % reduce hyphenation
\tolerance=1000
\doublespacing
\renewcommand\bibname{References}


\author{Yi Liu (yxl105[at]bham.ac.uk)\\
  \scriptsize{\url{https://github.com/YiLiu6240/bham_ECONG33_Regulation-Supervision}}}
\title{G33D - Class Note 4\\Basel II: Capital Regulation for Credit Risk}
\date{\today}

\begin{document}
\maketitle

There are three approaches to calculate regulatory capital for credit risk: the
standardised approach, the foundational IRB approach, and the advanced IRB
approach.

The intuition is to give banks the incentive to develop their own risk
management skill to enjoy lower capital from the more sophisticated approach
(\textit{why and how?}).

\section*{Standardised Approach}

The creditworthiness of the counterparties are now taken into account (in the
form of external credit ratings).

\begin{table}
  \begin{threeparttable}

    \caption{Risk Weights as a Percent of Principal for Exposures to
      Countries, Banks, and Corporations Under Basel II’s Standardised Approach \\
      \centerline{\footnotesize{
          % Hull, Chap. 12, pp.270
          \citep[Chap.12, pp.270]{hull2012risk} }} }

    \begin{tabular}{lp{1.5cm}p{1.5cm}p{1.5cm}p{1.5cm}p{1.5cm}p{1.5cm}p{1.5cm}}
      \toprule
      & AAA to AA- & A+ to A- & BBB+ to BBB- & BB+ to BB- & B+ to B- & Below B- & Unrated \\
      \midrule
      Country* & 0 & 20 & 50 & 100 & 100 & 150 & 100 \\
      Banks** & 20 & 50 & 50 & 100 & 100 & 150 & 50 \\
      Corporations & 20 & 50 & 100 & 100 & 150 & 150 & 100 \\
      \bottomrule
    \end{tabular}

    \begin{tablenotes}
    \item \footnotesize{ *Includes exposures to the country’s central bank }
    \item \footnotesize{ **National supervisors have options of incorporations }
    \end{tablenotes}

  \end{threeparttable}
\end{table}

The risk weight for retail lending: 75\%; for residential mortgage: 35\%; for
commercial real estate: 100\%.

\section*{The IRB Approach}

Intuition: The regulatory capital is to cover unexpected loss only. Expected
loss is to be covered by the pricing of risks.

\subsection*{Corporate, Sovereign, and Bank Exposures}

The risk-weighted assets are calculated as
\begin{align*}
  RWA & = 12.5 \times EAD \times LGD \times (WCDR - PD) \times MA \\
  \textrm{where} \\
  \rho & = 0.12(1 + e^{-50 \times PD}) \\
  WCDR & = N[\frac{ N^{-1} (PD) +\sqrt{\rho} N^{-1} (0.999) }{ \sqrt{ 1 - \rho } }] \\
  b & = [0.11852 - 0.05478 \times ln(PD)]^{2} \\
  MA & = \frac{ 1 + (M-2.5) \times b }{ 1 - 1.5 \times b }
\end{align*}

\(N(.)\) and \(N-1(.)\) can be calculated using NORM.S.DIST and NORM.S.INV
functions in Excel respectively(you can also use a normal distribution table).

\subsection*{Retail Exposures}

For residential mortgages, \(\rho\) is set to 0.15.

For other retail exposures, the risk-weigthed assets are calculated as
\begin{align*}
  RWA & = 12.5 \times EAD \times LGD \times (WCDR - PD) \\
  \textrm{where} \\
  \rho & = 0.03 + 0.13 e^{-35 \times PD}
\end{align*}


\FloatBarrier
\bibliographystyle{jmr}
\bibliography{G33D_bib}

\end{document}
