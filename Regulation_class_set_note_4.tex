\documentclass[12pt]{article} \usepackage[utf8]{inputenc}
\usepackage[]{geometry}
\usepackage{setspace}
\usepackage{pdflscape}
\usepackage{amsmath}
\usepackage{amsfonts}
\usepackage{amssymb}
\usepackage{sectsty}
\usepackage[titles,subfigure]{tocloft} % Alter the style of the Table of Contents
\usepackage[titletoc,toc,title]{appendix}
\usepackage{tabulary}
\usepackage{booktabs} % for much better looking tables
\usepackage[]{threeparttable}
\usepackage{array} % for better arrays (eg matrices) in maths
\usepackage{verbatim} % adds environment for commenting out blocks of text & for better verbatim
\usepackage{subfig} % make it possible to include more than one captioned figure/table in a single float
\usepackage{longtable}
\usepackage{multirow}
\usepackage{graphicx} % support the \includegraphics command and options
\usepackage{placeins} % clear floats
\usepackage[round]{natbib}
\usepackage{hyperref}

\hyphenpenalty=5000 % reduce hyphenation
\tolerance=1000
\doublespacing
\renewcommand\bibname{References}


\author{Yi Liu (yxl105[at]bham.ac.uk)\\
  \scriptsize{\url{https://github.com/YiLiu6240/bham_ECONG33_Regulation-Supervision}}}
\title{G33D - Class Note 4\\Basel II: Capital Regulation for Credit Risk}
\date{\today}

\begin{document}
\maketitle

There are three approaches to calculate regulatory capital for credit risk: the
standardised approach, the foundational IRB approach, and the advanced IRB
approach.

The intuition is to give banks the incentive to develop their own risk
management skill to enjoy lower capital from the more sophisticated approach
(\textit{why and how?}).

\section*{Standardised Approach}

The creditworthiness of the counterparties are now taken into account (in the
form of external credit ratings).

\begin{table}
  \begin{threeparttable}

    \caption{Risk Weights as a Percent of Principal for Exposures to
      Countries, Banks, and Corporations Under Basel II’s Standardised Approach \\
      \centerline{\footnotesize{
          % Hull, Chap. 12, pp.270
          \citep[Chap.12, pp.270]{hull2012risk} }} }

    \begin{tabular}{lp{1.5cm}p{1.5cm}p{1.5cm}p{1.5cm}p{1.5cm}p{1.5cm}p{1.5cm}}
      \toprule
      & AAA to AA- & A+ to A- & BBB+ to BBB- & BB+ to BB- & B+ to B- & Below B- & Unrated \\
      \midrule
      Country* & 0 & 20 & 50 & 100 & 100 & 150 & 100 \\
      Banks** & 20 & 50 & 50 & 100 & 100 & 150 & 50 \\
      Corporations & 20 & 50 & 100 & 100 & 150 & 150 & 100 \\
      \bottomrule
    \end{tabular}

    \begin{tablenotes}
    \item \footnotesize{ *Includes exposures to the country’s central bank }
    \item \footnotesize{ **National supervisors have options of incorporations }
    \end{tablenotes}

  \end{threeparttable}
\end{table}

The risk weight for retail lending: 75\%; for residential mortgage: 35\%; for
commercial real estate: 100\%.

\section*{The IRB Approach}

Intuition: The regulatory capital is to cover unexpected loss only. Expected
loss is to be covered by the pricing of risks.

\subsection*{Corporate, Sovereign, and Bank Exposures}

The risk-weighted assets are calculated as
\begin{align*}
  RWA & = 12.5 \times EAD \times LGD \times (WCDR - PD) \times MA \\
  \textrm{where} \\
  \rho & = 0.12(1 + e^{-50 \times PD}) \\
  WCDR & = N[\frac{ N^{-1} (PD) +\sqrt{\rho} N^{-1} (0.999) }{ \sqrt{ 1 - \rho } }] \\
  b & = [0.11852 - 0.05478 \times ln(PD)]^{2} \\
  MA & = \frac{ 1 + (M-2.5) \times b }{ 1 - 1.5 \times b }
\end{align*}

\(N(.)\) and \(N-1(.)\) can be calculated using NORM.S.DIST and NORM.S.INV
functions in Excel respectively(you can also use a normal distribution table).

\subsection*{Retail Exposures}

For residential mortgages, \(\rho\) is set to 0.15.

For other retail exposures, the risk-weigthed assets are calculated as
\begin{align*}
  RWA & = 12.5 \times EAD \times LGD \times (WCDR - PD) \\
  \textrm{where} \\
  \rho & = 0.03 + 0.13 e^{-35 \times PD}
\end{align*}

\newpage
\appendix
\section*{Appendix}
\subsection*{Worst Case Default Rate (\(WCDR\))}

Relevant information can be found in \citet[Chapter 11.5]{hull2012risk} and
\citet[Chapter 3.3]{trueck2009rating}.

Suppose we have the change in the value of an asset that follows a standarised
normal distribution, \(Z \sim N(0, 1)\), which can be further decomposed
\citep{belkin1998one} into two components: a systematic component (i.e. the
market condition) \(X\), and an idiosyncratic component \(\epsilon\),
\[ Z = \sqrt{\rho} X = \sqrt{1 - \rho} \epsilon. \]

where \(X \sim N(0, 1)\), \(\epsilon \sim N(0, 1)\) and \(\sqrt{\rho}\) denotes
the sensitivity of the asset against the systematic risk \(X\).

Denote \(c\) to be the critical value of ``the bankruptcy state'' in terms of
asset changes, and \(N\) to be the \(c.d.f\) of \(N(0, 1)\), the
``unconditional'' probability of default can then be expressed as
\[ Pr(Z < c) = N(c) = PD. \]
Or equivalently,
\[ c = N^{-1}(PD). \]
The probability of default, ``conditional'' on the
systematic risk at a specific level \(x\), is then
\begin{align*}
  Pr(Z < c | X = x) & = Pr \left( (\sqrt{\rho} X + \sqrt{1 - \rho} \epsilon) \right) \\
                    & = Pr \left( (\epsilon < \frac{C - \sqrt{\rho} X}{\sqrt{1 - \rho}} | X = x) \right) \\
                    & = N \left( \frac{c - \sqrt{\rho} x}{\sqrt{1 - \rho}} \right)
\end{align*}

In the context of Basel II, we measure \(WCDR\) as the probability of default of
the asset, conditional on the market condition at its \(0.001^{th}\) quantile
level. Therefore,
\begin{align*}
  WCDR & = N \left( \frac{N^{-1}(PD) - \sqrt{\rho} N^{-1}(0.001)}{\sqrt{1 - \rho}} \right) \\
       & = N \left( \frac{N^{-1}(PD) + \sqrt{\rho} N^{-1}(0.999)}{\sqrt{1 - \rho}} \right)
\end{align*}

\FloatBarrier
\bibliographystyle{jmr}
\bibliography{G33D_bib}

\end{document}
