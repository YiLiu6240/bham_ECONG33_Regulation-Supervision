\documentclass[12pt]{article}
\usepackage[utf8]{inputenc}
\usepackage[]{geometry}
\usepackage{setspace}
\usepackage{pdflscape}
\usepackage{amsmath}
\usepackage{amsfonts}
\usepackage{amssymb}
\usepackage{sectsty}
\usepackage[titles,subfigure]{tocloft} % Alter the style of the Table of Contents
\usepackage[titletoc,toc,title]{appendix}
\usepackage{tabulary}
\usepackage{booktabs} % for much better looking tables
\usepackage[]{threeparttable}
\usepackage{array} % for better arrays (eg matrices) in maths
\usepackage{verbatim} % adds environment for commenting out blocks of text & for better verbatim
\usepackage{subfig} % make it possible to include more than one captioned figure/table in a single float
\usepackage{longtable}
\usepackage{multirow}
\usepackage{graphicx} % support the \includegraphics command and options
\usepackage{placeins} % clear floats
\usepackage[round]{natbib}
\usepackage{hyperref}

\hyphenpenalty=5000 % reduce hyphenation
\tolerance=1000
\doublespacing
\renewcommand\bibname{References}


\author{Yi Liu (yxl105[at]bham.ac.uk)\\\scriptsize{\url{https://github.com/YiLiu6240/bham_ECONG33_Regulation-Supervision}}}
\date{\today}
\title{G33D - Class Note 5\\Basel II: Operational Risk, and Basel III}

\begin{document}
\maketitle
\tableofcontents

\section*{Opeartional Risk of Basel II}

Defining operational risk by the Basel Committee: ``...the risk of loss resulting from inadequate or failed internal processes, people and systems or from external events''.

Four categories of causal factors:

\begin{enumerate}
    \item Internal processes;
    \item Human resources;
    \item Systems;
    \item External events;
\end{enumerate}

Note: this definition includes legal risk but excludes strategic and reputational risk.

Examples of Operational Rsik: available in your lecture notes. One of the most famous examples is the collapse of Barings Bank from rogue trading.

Credit risks and market risks are easier to quantify (such as the parameters in IRB approach, or the VaR technique for market risk). It is difficult for a firm to quantify its operational risk.

The basic indicator approach and standardised approach are based on calculating the gross operating income (GOI) of a period. The capital requirement is then calculated by appying a risk factor to the GOI.


\subsection{Basic Indicator Approach}

\begin{align*}
    K_{OR} = \alpha \frac{ \sum_{i=1}^{3} Max(0, GOI_{t-i}) }{ N }
\end{align*}

{\footnotesize where N is the number of years, of the last three, in which GOI was positive}

For BIA, the risk factor $ \alpha $ is fixed at 15\%.


\subsection{Standardised Approach}

\begin{align*}
    K_{OR} = \sum_{i=1}^{3} \frac{ Max(\sum_{j=1}^{8} \beta_j {GOI}_{j, t-i}, 0) }{ 3 }
\end{align*}

For standardised approach, the risk factor $ \beta $ varies with 8 business lines

\begin{table}
    \begin{threeparttable}

        \caption[Risk Factors in Basel II Standardised Approach]{
            \textbf{Risk Factors in Basel II Standardised Approach} \\
            \centerline{\footnotesize{
                % Basel Committe(2006)
                \defcitealias{Basel2006Supervision}{Basel Committee, 2006}
                \citepalias{Basel2006Supervision}
            }}
        }

        \begin{tabular}{lll}
            \toprule
            $ J $ & Business Line            & Beta Factor \\
            1     & Corporate Finance        & 18\%        \\
            2     & Trading and Sales        & 18\%        \\
            3     & Retail Banking           & 12\%        \\
            4     & Commercial Banking       & 15\%        \\
            5     & Payments and Settlements & 18\%        \\
            6     & Agency Services          & 15\%        \\
            7     & Asset Management         & 12\%        \\
            8     & Retail Brokerage         & 12\%        \\
            \bottomrule
        \end{tabular}

        \begin{tablenotes}
            \item
        \end{tablenotes}

    \end{threeparttable}
\end{table}
\FloatBarrier

\section*{Stressed $ VaR $($ sVaR $) of Basel 2.5}

Historical simulation: For $ VaR $, the historical period is one to four years; For $ sVaR $, the historical period is 250-day of stressed market conditions (please refer to
 Hull, Chap 14
 for historical simulation approach).

Capital calculation:

\begin{align*}
    Max({VaR}_{t-1}, m_c \times {VaR}_{avg}) + Max({sVaR}_{t-1}, m_s \times {sVaR}_{avg})
\end{align*}

\section*{Capital Requirement of Basel III}
Three levels: T1 capital, T1+T2 capital, ``capital conservation buffer''
\begin{itemize}

    \item T1: 6\% (up from 4\%), of which core T1 4.5\%
    \item T1+T2: 8\%
    \item capital conservation buffer: 2.5\% core T1

\end{itemize}
\begin{figure}
  \centering
  \includegraphics[scale=0.8]{output/Basel_III_capital.pdf}
\end{figure}

\FloatBarrier

\section*{Leverage Ratio of Basel III}

Ratio of capital to total exposure
Minimum of 3\%

\section*{Liquidity of Basel III: Net Stable Funding Ratio}

\begin{align*}
    \frac{ \textit{Amount of stable funding} }{ \textit{Required amount of stable funding} } \geq 100\%
\end{align*}

Each items in the numerator/the denominator is weighted by a risk factor. The available stable funding factor (ASF) for the numerator and the required stable funding (RSF) for the denominator.

\begin{table}
    \begin{threeparttable}

        \caption[ASF and RSF factors for Net Stable Funding Ratio]{
            \textbf{ASF and RSF factors for Net Stable Funding Ratio} \\
            \centerline{\footnotesize{
                \citep[Chap.13, pp.293-294]{hull2012risk}
            }}
        }

        \begin{tabular}{lp{12cm}}
            \toprule
            \textbf{ASF Factor} & Category                                                                                                                                                                                                          \\ \hline
            100\%               & Tier 1 and Tier 2 capital; Preferred stock and borrowing with a remaining maturity greater than one year                                                                                                          \\ \hline
            90\%                & ``Stable'' demand deposits and term deposits with remaining maturity less than one year provided by retail or small business customers                                                                            \\ \hline
            80\%                & ``Less Stable'' demand deposits and term deposits with remaining maturity less than one year provided by retail or small business customers                                                                       \\ \hline
            50\%                & Wholesale demand deposits and term deposits with remaining maturity less than one year provided by nonfinancial corporates, sovereigns, central banks, multilateral development banks, and public sector entities \\ \hline
            0\%                 & All other liability and equity categories                                                                                                                                                                         \\ \hline
            \multicolumn{2}{l}{  }                                                                                                                                                                                                                  \\ \hline
            \textbf{RSF Factor} & Category                                                                                                                                                                                                          \\ \hline
            0\%                 & Cash; Short-term instruments, securities, loans to financial entities if they have a residual maturity of less than one year                                                                                      \\ \hline
            5\%                 & Marketable securities with a residual maturity greater than one year if they are claims on sovereign governments or similar bodies with a 0\% risk weight                                                         \\ \hline
            20\%                & Corporate bonds with a rating of AA- or higher and a residual maturity greater than one year; Claims on sovereign governments or similar bodies with a risk weight of 20\%                                        \\ \hline
            50\%                & Gold, equity securities, bonds rated A+ to A-                                                                                                                                                                     \\ \hline
            65\%                & Residential mortgages                                                                                                                                                                                             \\ \hline
            85\%                & Loans to retail and small business customers with a remaining maturity less than one year                                                                                                                         \\ \hline
            100\%               & All other assets                                                                                                                                                                                                  \\
            \bottomrule
        \end{tabular}

        \begin{tablenotes}
            \item
        \end{tablenotes}


    \end{threeparttable}
\end{table}

\FloatBarrier
\bibliographystyle{jmr}
\bibliography{G33D_bib}

\end{document}
